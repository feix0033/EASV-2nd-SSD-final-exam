% ============================================
% Document Class and Basic Setup
% ============================================
\documentclass[11pt]{article}

% ============================================
% Package Imports
% ============================================
% Geometry and Layout
\usepackage[margin=2.5cm]{geometry}
\usepackage{xcolor}

% Fonts and Encoding
\usepackage{iftex}
\ifPDFTeX
  \usepackage[T1]{fontenc}
  \usepackage[utf8]{inputenc}
  \usepackage{textcomp}
\else
  \usepackage{unicode-math}
  \defaultfontfeatures{Scale=MatchLowercase}
  \defaultfontfeatures[\rmfamily]{Ligatures=TeX,Scale=1}
\fi
\usepackage{lmodern}

% Typography
\IfFileExists{microtype.sty}{%
  \usepackage{microtype}
  \UseMicrotypeSet[protrusion]{basicmath}
}{}

% Paragraph Formatting
\makeatletter
\@ifundefined{KOMAClassName}{%
  \IfFileExists{parskip.sty}{%
    \usepackage{parskip}
  }{%
    \setlength{\parindent}{0pt}
    \setlength{\parskip}{6pt plus 2pt minus 1pt}
  }
}{%
  \KOMAoptions{parskip=half}
}
\makeatother

% Math
\usepackage{amsmath,amssymb}

% Utilities
\setlength{\emergencystretch}{3em}
\providecommand{\tightlist}{%
  \setlength{\itemsep}{0pt}\setlength{\parskip}{0pt}}
\setcounter{secnumdepth}{-\maxdimen}
% Headers and Footers
\usepackage{fancyhdr}
\usepackage{lastpage}

% Graphics
\usepackage{graphicx}

% TOC Management
\usepackage{etoolbox}

% ============================================
% Custom Title Page
% ============================================
\makeatletter
\renewcommand{\maketitle}{%
  \begin{titlepage}
    \thispagestyle{empty}
    \begin{center}
      \vspace*{3cm}

      % Title
      {\Huge\bfseries\@title\par}
      \vspace{2cm}

      % Subtitle and Course
      {\Large PB Software Development\par}
      \vspace{0.5cm}
      {\large Software Security\par}
      \vspace{3cm}

      % Date
      {\large\textit{Fall 2025}\par}
      \vfill

      % Author
      {\large\textit{Synopsis by Marco Gabel \& Fei Gu}\par}
      \vspace{1cm}

      % Logo (uncomment to use)
      % \includegraphics[width=0.3\textwidth]{logo.png}
    \end{center}
  \end{titlepage}
  \clearpage
}
\makeatother

% ============================================
% Headers and Footers Configuration
% ============================================
% Content pages style
\pagestyle{fancy}
\fancyhead[L]{\small\textit{Marco Gabel \& Fei Gu}}
\fancyhead[C]{\small\textit{EASV SSD}}
\fancyhead[R]{\small\textit{Exam Synopsis 2026}}
\fancyfoot[C]{\small\thepage\ af \pageref{LastPage}}
\fancyfoot[L]{}
\fancyfoot[R]{}
\renewcommand{\headrulewidth}{0.4pt}
\renewcommand{\footrulewidth}{0.4pt}

% Plain style (for title and TOC pages)
\fancypagestyle{plain}{%
  \fancyhf{}
  \fancyfoot[C]{\small\thepage\ af \pageref{LastPage}}
  \renewcommand{\headrulewidth}{0pt}
}

% ============================================
% Table of Contents Configuration
% ============================================
\AtEndDocument{%
  \addtocontents{toc}{\protect\clearpage}
}
% ============================================
% Hyperlinks and PDF Metadata
% ============================================
\usepackage{bookmark}
\IfFileExists{xurl.sty}{\usepackage{xurl}}{}
\urlstyle{same}
\hypersetup{
  pdftitle={Exam Synopsis -- Securing CI/CD Pipeline with SNYK, CodeQL, and Trivy},
  pdfauthor={Marco Gabel \& Fei Gu},
  pdfkeywords={SNYK, CodeQL, Trivy, DevSecOps, CI/CD Security, CRA},
  colorlinks=true,
  linkcolor=blue,
  filecolor=Maroon,
  citecolor=Blue,
  urlcolor=blue,
  pdfcreator={LaTeX via pandoc}
}

% ============================================
% Document Metadata
% ============================================
\title{Exam Synopsis -- Securing CI/CD Pipeline with SNYK, CodeQL, and Trivy}
\author{Marco Gabel \& Fei Gu}
\date{Fall 2025}

% ============================================
% Document Body
% ============================================
\begin{document}

% Generate title page
\maketitle

% Generate table of contents
\renewcommand*\contentsname{Table of Contents}
{
  \hypersetup{linkcolor=black}
  \setcounter{tocdepth}{3}
  \tableofcontents
}

% ============================================
% Content Sections
% ============================================

\section{1. Introduction / Motivation}\label{introduction-motivation}

% The motivation for picking this particular problem to work on should be
% clearly communicated in the synopsis. Explain why this problem is
% interesting for you to work and why it will be interesting for others to
% read about. Feel free to include any background information that that is
% required to understand the context of the problem.

% The synopsis is a short documentation of your project work.

\section{2. Problem Statement}\label{problem-statement}

This project investigates how a CI/CD pipeline can be secured using
automated analysis tools such as SNYK (Software Composition Analysis),
CodeQL (Static Application Security Testing) and Trivy (Container
Scanning). The goal is to identify vulnerabilities during the build
process and assess how these security controls support the Cyber
Resilience Act requirements for secure software development and
supply-chain protection.

\section{3. Implementation}\label{inplementation}

\subsection{3.1 Pipeline architecture}\label{pipeline-architecture}

\subsection{3.2 Tool integration details}\label{tool-integration-details}

\subsection{3.3 Security findings}\label{security-findings}

\subsection{3.4 CRA compliance mapping}\label{cra-compliance-mapping}

% In this section you should describe how you intend to analyze and solve
% the problem. Refer to relevant litterature or videos that contains
% information that will help you solve the problem.

% Describe in more detail how you will solve the problem: what will you
% implement? What will you measure? How will you evaluate your
% measurements?

% And finally: What will confirm or reject your hypothesis?

\section{4. Analysis \& Results}\label{analysis-results}

% This is the meat of your synopsis.

% You have to outline what you have done in this section. Document the
% effort you've put into analyzing the problem and also the work you've
% done building and implementing a solution.

% This isn't your GitHub repository. You should not include all the code
% you've written. Instead, you should include the most important parts of
% your work.

% What is important? Parts of the code that are a result of your analysis
% and the choices you've made. Connect theory and analysis to how the code
% is written.

% Demonstrate that you understand the theory and that you can apply it to
% solve a problem. Don't expect and plan for creating a ``product''. Write
% only enough code, outside your main objective and focus on your problem.

\section{5. Conclusion}\label{conclusion}

% Summarize.

% This is your chance to briefly summarize what you've done throughout the
% project. Focus on the most important parts and takeways from your work.
% And finally, what did you find out? Did you confirm or reject your
% hypothesis?

\end{document}
